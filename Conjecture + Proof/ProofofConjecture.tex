\documentclass{article}
\usepackage[utf8]{inputenc}
\usepackage{amssymb}
\usepackage{amsmath}

\title{Proof of Conjecture}
\author{Anay Aggarwal}
\date{2021}

\begin{document}

\maketitle

\section*{Conjecture}
The $n\times n$ matrix $\mathbb{A}$ can be row-reduced to the $n\times n$ identity matrix $\mathbb{I}$ in $n^2$ moves.
\section*{Construction}
Partition $\mathbb{A}$ into $n$ column vectors. Define the index of each vector as the column number of that vector. Starting with the vector with the smallest index, go down the vector, then move on to the next vector. At each $\mathbb{A}_{i,k}$ perform
 $$R_i\to \begin{cases} 
      R_i-\frac{\mathbb{A}_{i,k}}{\mathbb{A}_{i+1,k}}R_{i+1} & \text{ if }i<k \\
      R_i\cdot \frac{1}{
      \mathbb{A}_{i,k}}& \text{ if } i=k \\
      R_i-\mathbb{A}_{i,k}R_k & \text{ if } i>k
   \end{cases}$$
\section*{Proof that the elements match those of the identity matrix}
Case 1: $i<k$.
\par We need $\mathbb{A}_{i,k}=0$. Notice that in the specific column, \par $R_{i+1}=A_{i+1, k}$. And since $\frac{\mathbb{A}_{i,k}}{\mathbb{A}_{i+1,k}}\cdot\mathbb{A}_{i+1,k}=-\mathbb{A}_{i,k}$, we're done. 
\\ \\ Case 2: $i=k$
\par Clearly, $\mathbb{A}_{i,k}\cdot \frac{1}{\mathbb{A}_{i,k}}=1$, as desired.
\\ \\ Case 3: $i>k$
\par Notice that $\mathbb{A}_{k,k}=1$, since we have already worked on the elements in row \par $k$. So $$\mathbb{A}_{i,k}-\mathbb{A}_{i,k}\mathbb{A}_{k,k}=\mathbb{A}_{i,k}-\mathbb{A}_{i,k}=0,$$
\par as desired.
\par 
\section*{Proof that each move preserves all of the other elements}
By Induction, it suffices to prove that all $\mathbb{A}_{i,j}|j<k$ are preserved. 
\\ \\ Case 1: $i<k$.
\par Notice that $\mathbb{A}_{a_1, a_2}=0\forall a_2<k, a_1\ne a_2$. Therefore, unless $i+1=k$, we are \par subtracting $c\cdot 0=0$, for constant $c$, from each element, thus preserving \par them. \par So it suffices to show that this works for $i+1=k$. Notice that the matrix \par is now
$$\begin{vmatrix}1 & 0 & 0 & 0 & ... & 0\\ 0 & 1 & 0 & 0 & ...  & 0\\ ... & ... & ... & ... & ... & ...\\ 0 & ... & 1 &\mathbb{A}_{i,k}&...&0\\ \boxed{0} & ... & \boxed{0} & 1 & ... & 0\end{vmatrix}.$$
\par Since all boxed elements are $0$, we're done with this case. 
\\ \\ Case 2: $i=k$
\par Notice that all previous elements are $0$, so multiplying them by a constant \par will preserve them.
\\ \\ Case 2: $i>k$
\par Clearly, it suffices to show that $\mathbb{A}_{k,j}=0\forall j\in\mathbb{Z}_{<k}$. This follows immediately \par from our inductive hypothesis.
\end{document}

